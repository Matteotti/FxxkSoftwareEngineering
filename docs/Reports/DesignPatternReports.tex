\documentclass[24pt,a4paper]{article}% 文档格式
\usepackage{ctex,hyperref}% 输出汉字
\usepackage{times}% 英文使用Times New Roman
%%%%%%%%%%%%%%%%%%%%%%%%%%%%%%%%%%%%%%%%%%%%%%%%%%%%%%%%
\title{\fontsize{18pt}{27pt}\selectfont% 小四字号,1.5倍行距
	{\heiti% 黑体 
		一种\LaTeX 模板}}% 题目
%%%%%%%%%%%%%%%%%%%%%%%%%%%%%%%%%%%%%%%%%%%%%%%%%%%%%%%%
\date{}% 日期(这里避免生成日期)
%%%%%%%%%%%%%%%%%%%%%%%%%%%%%%%%%%%%%%%%%%%%%%%%%%%%%%%%
\usepackage{amsmath,amsfonts,amssymb}% 为公式输入创造条件的宏包
%%%%%%%%%%%%%%%%%%%%%%%%%%%%%%%%%%%%%%%%%%%%%%%%%%%%%%%%
\usepackage{graphicx}% 图片插入宏包
\usepackage{subfigure}% 并排子图
\usepackage{float}% 浮动环境,用于调整图片位置
\usepackage[export]{adjustbox}% 防止过宽的图片
\usepackage{caption}
%%%%%%%%%%%%%%%%%%%%%%%%%%%%%%%%%%%%%%%%%%%%%%%%%%%%%%%%
\usepackage{url}% 超链接
\usepackage{bm}% 加粗部分公式
\usepackage{listings}
\usepackage{xcolor}
\lstset{
    language=Java,
    basicstyle=\ttfamily,
    keywordstyle=\color{blue},
    commentstyle=\color{gray},
    stringstyle=\color{orange}
}
\usepackage{multirow}
\usepackage{booktabs}
\usepackage{epstopdf}
\usepackage{epsfig}
\usepackage{longtable}% 长表格
\usepackage{supertabular}% 跨页表格
\usepackage{algorithm}
\usepackage{algorithmic}
\usepackage{changepage}% 换页
\usepackage{listings}% 插入代码段
%%%%%%%%%%%%%%%%%%%%%%%%%%%%%%%%%%%%%%%%%%%%%%%%%%%%%%%%
\usepackage[left=2.50cm,right=2.50cm,top=2.80cm,bottom=2.50cm]{geometry}% 页边距设置
\renewcommand{\baselinestretch}{1.5}% 定义行间距(1.5)
\renewcommand{\contentsname}{\normalfont \kaishu \Huge 目录}% 定义目录两字的格式

\usepackage{subfigure}% 有关设置目录引导点的宏包
\usepackage[subfigure]{tocloft}
\renewcommand{\cftsecleader}{\cftdotfill{\cftdotsep}} % 给 sections加点
\newcommand\mydot[1]{\scalebox{#1}{.}}
\renewcommand\cftdot{\mydot{0.8}} % change the size of dots
\renewcommand\cftdotsep{3} % change the space between dots

\hypersetup{
colorlinks=true,
linkcolor=black
}% 设置链接的颜色,防止目录出现红框
%%%%%%%%%%%%%%%%%%%%%%%%%%%%%%%%%%%%%%%%%%%%%%%%%%%%%%%%

\begin{document}% 以下为正文内容
\begin{titlepage}
    \centering
    \includegraphics[width=0.3\textwidth]{image/zju_logo.png}\par\vspace{5cm}
    {\huge\songti 设计模式报告\par}
    \vspace{1cm}
    {\Large\itshape 第三小组\par}
    \vspace{7cm}

    \vfill
    {\large \today\par}
\end{titlepage}
\newpage

\begin{center}
    \kaishu
    \tableofcontents
    \setcounter{page}{0}
    \thispagestyle{empty} % 设置目录页的页脚为空
\end{center}
\newpage

\section*{\songti 第一部分:外观模式}
\addcontentsline{toc}{section}{1.外观模式}
\subsection*{\songti 1.1概念}
\addcontentsline{toc}{subsection}{1.1概念}
外观模式是一种常用的软件设计模式,属于结构型模式的一种。它为一个子系统中的一组接口提供一个简洁的,统一的界面,用来掩盖系统的复杂性,使得子系统中的一个接口更加容易使用。这可以通过定义一个包含子系统所有可能功能的单一类来实现的。
\subsection*{\songti 1.2意图}
\addcontentsline{toc}{subsection}{1.2意图}
外观模式的主要意图为子系统中一组接口提供一个统一的界面,从而降低客户端与复杂子系统之间的交互难度,通过创建统一的界面,来控制客户端和复杂系统的交互。其核心目的在于隐藏子系统的实现细节,让子系统使用起来更简便,从而减轻客户端的使用负担。

在实际的应用中,一个复杂系统经常拥有大量的组件与底层的模块。如果客户端直接与这些复杂组件和模块交互,将会导致代码耦合度高,维护困难。在这个时候,使用外观模式将这些复杂的操作封装起来,变得尤为重要,因为通过定义一个高层、简单易用的接口库,为客户端提供一个统一的界面,使客户端能够以一种更加理解和更易使用的方式与子系统进行交流和操作。这个模式把复杂的业务逻辑和操作封装在一个简洁易用的“外观”接口之后,并通过系统内部的协调和调度来完成实际的业务工作。

这样不仅降低了子系统的复杂度,也降低了客户端使用子系统的学习成本,让客户端只需要学习接口的定义和功能即可;同时外观模式也在一定程度上提升了代码的可阅读性和可维护性。在子系统内部需要进行变化或升级时,客户端的代码不需要进行修改,由外观接口内部的子系统来完成这些更改升级即可。

总体而言,外观模式的意图是将一个复杂系统背后的许多复杂交互行为隐藏起来,面向客户端来提供一个简单易用、一致性的界面,从而减轻客户端的使用复杂度,使子系统的使用更加方便。
\subsection*{\songti 1.3结构和组成}
\addcontentsline{toc}{subsection}{1.3结构和组成}
外观模式主要由以下几部分组成:
\subsubsection*{\songti 1.3.1\ 外观类}
外观类是外观模式的核心,客户端通过调用该类来访问子系统,该类知道每个子系统的功能和责任,将客户端的请求代理给适当的子系统对象。
\subsubsection*{\songti 1.3.2\ 子系统类}
在外观模式中,可以有一个或多个子系统。每个子系统并不是单一的类,而是由多个类组成。在子系统里,每个类都有各自的功能,这些功能之间可以相互交互,共同协作完成任务。但对外部客户端来说,这些交互被隐藏起来不可见,客户端只可以通过外观类来访问子系统的各个功能。
\subsubsection*{\songti 1.3.3\ 客户端}
在外观模式中,客户端通过调用外观类中的功能来完成对于整个子系统的访问,而无需关心子系统的内部工作细节。
\subsection*{\songti 1.4优点和缺点}
\addcontentsline{toc}{subsection}{1.4优点和缺点}

\subsubsection*{\songti 1.4.1\ 优点}
\begin{itemize}
    \item \textbf{简化客户端的复杂度}:外观模式隐藏了系统的复杂性,使客户端代码更加简洁和易于使用。
    \item \textbf{解耦系统各个部分}:外观模式可以将系统各个部分解耦,使它们能够独立变化。这样,系统的某个部分的变化不会影响到其他部分,从而便于各个部分的更改。
    \item \textbf{提高系统的可维护性}:外观模式将系统的复杂性隐藏在了一个单独的外观类中,使代码更加清晰和易于维护。
    \item \textbf{提高系统的安全性}:外观模式可以控制客户端能够访问系统的部分,从而提高系统的安全性。
\end{itemize}

\subsubsection*{\songti 1.4.2\ 缺点}
\begin{itemize}
    \item \textbf{缺乏细粒度控制}:不外观模式提供了一组简单的接口,用于访问系统的各个部分。但是如果需要更细粒度的控制,可能需要修改外观类或者直接访问系统的各个部分。
    \item \textbf{增加复杂性}:外观模式需要引入一个额外的外观类,如果客户端需要与子系统进行大量交互,可能会增加系统的复杂性。
    \item \textbf{影响性能}如果外观类的实现不够高效,可能会影响系统的性能。
    \item \textbf{提高系统的安全性}:外观模式可以控制客户端能够访问系统的部分,从而提高系统的安全性。
\end{itemize}

\subsection*{\songti 1.5应用场景}
\addcontentsline{toc}{subsection}{1.5应用场景}
\begin{enumerate}
    \item 设计一个简单的方法以便于客户端调用复杂的系统。例如,为复杂的计算机系统设计一个启动按钮。在客户按下启动按钮后,计算机开始进行内部检查、完成系统的加载、进行进程的调度等一系列操作,而对于客户来说,仅需要点击一个按钮就可以了,而不用关心后面的复杂过程。
    \item 当需要定义系统的多级访问权限时,可以通过在系统一级接口的基础上封装一层访问接口,控制用户对于某些高级接口的访问。例如在网络服务器的设计中,在不同级别的用户接口上再封装一层外观接口,从而控制用户对于网络资源的访问。
    \item 将子系统和客户端解耦,使得他们之间的依赖关系降低,同时也使子系统更易于独立开发和维护,适合在大型系统设计时使用。
\end{enumerate}

\subsection*{\songti 1.6工业界的实际应用}
\addcontentsline{toc}{subsection}{1.6工业界的实际应用}
\begin{enumerate}
    \item \textbf{Java的Spring框架}:Spring框架提供了一个统一的架构,把复杂的事务管理、安全性、数据库访问等功能封装成为了一个统一的接口,使得用户可以非常方便地使用这些复杂功能。
    \item \textbf{数据库连接}:许多应用程序使用的数据库连接库,例如 Java 的JDBC,为复杂的 SQL 查询和数据库操作提供了简单的接口。通过这些接口,程序员不再需要直接编写 SQL 语句,只需要调用一些易于理解的方法和函数就能进行数据库的操作。
    \item \textbf{电商平台 API}:例如 Amazon、eBay 和 阿里巴巴 这样的电商平台,它们提供了一系列的API接口来帮助开发者进行商品上传、查询、售卖和物流管理等一系列操作。商家不需要关注具体的实现细节,只需要调用相应的接口就可以方便地进行交易。
    \item \textbf{GUI库}:在许多 GUI(图形用户界面)库中,例如Swing、Qt等,都使用了外观模式思想,他们为复杂的底层图形系统提供了简单易用的API接口,牵扯到窗口管理、事件处理等一系列复杂的操作,却只需要调用简单的接口函数。这大幅度降低了GUI编程难度,提高开发效率。
\end{enumerate}

\subsection*{\songti 1.7本项目中的应用}
\addcontentsline{toc}{subsection}{1.7本项目中的应用}
本项目中,外观类主要运用在工具类和底层数据的使用上。对于一些需要频繁使用的工具类和算法,例如ThreadLocal、Mail 和MD5,本项目的后端选择将其集成为工具类,供其他类使用。这样可以大大简化系统代码,并很好地解耦了高级类和低级方法。

一个例子如下
\begin{lstlisting}[language=Java]
@SuppressWarnings("all")
public class ThreadLocalUtil {
    // 提供ThreadLocal对象,
    private static final ThreadLocal THREAD_LOCAL = new ThreadLocal();
    // 根据键获取值
    public static <T> T get() {
        return (T) THREAD_LOCAL.get();
    }
    // 存储键值对
    public static void set(Object value) {
        THREAD_LOCAL.set(value);
    }
    // 清除ThreadLocal 防止内存泄漏
    public static void remove() {
        THREAD_LOCAL.remove();
    }
}
\end{lstlisting}
我们将提供线程局部变量的 ThreadLocal 包装为一个外观类,从而简化了其他类对其的使用。

此外,我们还将针对数据库的操作使用外观模式包装,将数据库操作与业务逻辑分离,使得业务逻辑代码更加清晰和可维护。

一个例子如下:
\begin{lstlisting}[language=Java]
@Mapper
public interface AccountMapper {
    @Update("update account set balance=#{balance} where acno=#{acno}")
    void updateAccount(int acno, double balance);
    
    @Select("select * from account where pano=#{pano}")
    Account findAccountByPano(int pano);
    
    @Select("select Balance from account where pano=#{pano}")
    double findBalanceByPano(int pano);
}
\end{lstlisting}
我们将一些必要的数据库操作包装成一个外观类,在外观类内部通过 @Select 注解指定对应的 SQL 查询语句,实现了对数据库表的操作,并将这些操作作为接口提供给其他类使用。

\newpage
\end{document}% 结束文档编辑,后面写啥都编译不出来

%%%%%%%%%%%%%%%%%%%%%草稿%%%%%%%%%%%%%%%%%%%%%
% 插入图片 %
\begin{figure}[H]
    \centering
    \includegraphics[width=1\textwidth]{images/*.png}
    \caption*{图*.键入标题...}
\end{figure}
%%%%%%%%%%%%%%%%%%%%%%%%%%%%%%%%%%%%%%%%%%%%%